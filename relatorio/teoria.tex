%---------------------------------------------------------------------
\section{Backstepping - Observador de ordem reduzida}

Este trabalho visa complementar o trabalho 7, modificando o observador completo
por um observador de ordem reduzida, tamb�m chamado de observador de Luenberger.
A formula��o te�rica passa pela ideia geral de um observador de ordem reduzida,
exemplifica para o caso do sistema de segunda ordem deste trabalho e, por
�ltimo, desenvolvemos o algoritmo backstepping para este observador e caso
$n=2$, $n^*=2$.

Considere uma planta descrita pelo seguinte sistema em espa�o de estados:
\begin{align}
\dot{x} &= Ax + Bu\\
\nonumber y &= Cx
\label{eq:planta}
\end{align}

Suponha que os primeiros $m$ estados podem ser obtidos diretamente pela medida
da sa�da do sistema, ou seja, o sistema pode ser paticioando:

\begin{align}
\dot{x}_1 &= A_{11}x_1 + A_{12}x_2 + B_1u \\
\nonumber \dot{x}_2 &= A_{21}x_1 + A_{22}x_2 + B_2u \\
\nonumber y &= C_1x_1
\end{align}

e $x_1 = C_1^{-1}y$. Um observador de ordem reduzida pode ser usado para estimar
os $x_2 \in \mathbb{R}^{n-m}$ estados faltantes. Define-se:

\begin{align}
\chi = x_2 + Ny
\label{eq:csi}
\end{align}

Pode-se demonstrar que a din�mica de $\chi$ � descrita como:

\begin{align}
\chi &= Q\chi + Ry + Su \\
\nonumber Q &= A_{22} + NC_1A_{12}\\
\nonumber R &= -QN + (A_{21}+NC_1A_{11})C_1^{-1}\\
\nonumber S &= B_2 + NC_1B_1
\end{align}

Derivando a equa��o~\ref{eq:csi}, obtemos:

\begin{align}
\dot{\chi} &= \dot{x}_2 + NC_1\dot{x}_1 \\
\nonumber &= (A_{21}x_1 + A_{22}x_2 + B_2u) + NC_1(A_{11}x_1 + A_{12}x_2 +
B_1u)\\
\nonumber &= (A_{22} + NC_1A_{12})x_2 + (A_{21}+NC_1A_{11})x_1 + (B_2+NC_1B_1)u
\\
\nonumber &= (A_{22} + NC_1A_{12})x_2 + (A_{22}+NC_1A_{12})Ny -
(A_{22}+NC_1A_{12})Ny + (A_{21}+NC_1A_{11})x_1 + (B_2+NC_1B_1)u \\
\nonumber &= (A_{22} + NC_1A_{12})(x_2+Ny) - (A_{22}+NC_1A_{12})Ny +
(A_{21}+NC_1A_{11})C_1^{-1}y + (B_2+NC_1B_1)u \\
\nonumber &= Q\chi + \left[-QN + (A_{21}+NC_1A_{11})C_1^{-1}\right]y +
Su\\
\nonumber &= Q\chi + Ry + Su
\end{align}

Neste trabalho, consideramos o sistema:

\begin{align}
\label{eq:planta2}
\dot{x}_1 &= x_2 - a_1y\\
\nonumber \dot{x}_2 &= k_p\,u - a_0y
\end{align}

onde os par�metros $a_1$, $a_0$ e $k_p$ s�o desconehcidos. Para esta formula��o
apenas a sa�da do sistema $y$ est� dispon�vel, portanto $x_2$ n�o � conhecido e
deve ser estimado. Podemos reescrever o sistema \ref{eq:planta2}:

\begin{align}
\nonumber \dot{x} &= Ax - F(y,u)^\intercal\theta \\
A &= 
\begin{bmatrix}
0 & 1\\
0 & 0
\end{bmatrix}, F(y,u)^\intercal = 
\nonumber \begin{bmatrix}
B(u) & \Phi(y)
\end{bmatrix}, \Phi(y) = 
\begin{bmatrix}
-y & 0\\
0 & -y
\end{bmatrix}, B(u) = 
\begin{bmatrix}
0\\
u
\end{bmatrix}, \theta =
\begin{bmatrix}
k_p \\
a_1 \\
a_0
\end{bmatrix} \\
\nonumber y &= e_1^\intercal x \\
\nonumber e_1 &= 
\begin{bmatrix}
1 \\
0
\end{bmatrix} \\
\end{align}

No trabalho 7, para estimar os estados, utilizamos os filtros abaixo:

\begin{align}
\label{eq:filtros2}
\dot{\xi} &= A_0\xi + ky \\
\nonumber \dot{\Omega}^\intercal &= A_0\Omega^\intercal + F^\intercal\\
\nonumber k &=
\begin{bmatrix}
k_1\\k_2
\end{bmatrix}, A_0 = A - ke_1^\intercal =  
\begin{bmatrix}
-k_1 & 1\\-k_2 & 0
\end{bmatrix}
\end{align}

Os valores de $k$ devem ser escolhidos de forma que $A_0$ seja Hurwitz. E, dessa
forma, o estado estimado pode ser escrito como:

\begin{align}
\hat{x} = \xi + \Omega^\intercal\theta
\label{eq:estimador}
\end{align}

Como no trabalho 7, verifica-se que a din�mica do estimador � igual �
din�mica da planta \ref{eq:planta2}.

Para o caso do observador de ordem reduzida, define-se:

\begin{align}
\chi = x_2 + Ny
\end{align}

E derivando, obtemos:

\begin{align}
\dot{\chi} &= (-a_0y + k_pu) + N(x_2-a_1y) \\
\nonumber &= Nx_2 - (a_0 + Na_1)y + k_pu \\
\nonumber &= N(\chi - Ny) - (a_0 + a_1)y + k_pu\\
\nonumber &= N\chi - N^2y + F^\intercal\theta\\
\nonumber F^\intercal &= \left[u \quad -Ny \quad -y\right], \theta =
\left[k_p \quad a_1 \quad a_0\right]^\intercal
\end{align}

Para o sistema de ordem reduzida, os filtros s�o:

\begin{align}
\label{eq:filtros3}
\dot{\xi} &= N\xi - N^2y \\
\nonumber \dot{\Omega}^\intercal &= N\Omega^\intercal + F^\intercal\\
\nonumber & N < 0
\end{align}

E o estado estimado ser�:

\begin{align}
\hat{\chi} &= \xi + \Omega^\intercal\theta \\
\dot{\hat{\chi}} &= \dot{\xi} + \Omega^\intercal\theta \\
\nonumber &= (N\xi - N^2y) + (N\Omega^\intercal + F^\intercal)\theta \\
\nonumber &= N(\xi+\Omega^\intercal\theta) - N^2y + F^\intercal\theta \\
\dot{\hat{\chi}} &= N\hat{\chi}-N^2y+F^\intercal\theta
\end{align}

Por�m, $\Omega$ � uma matriz e opta-se pela redu��o das ordens dos
filtros. Observe que $\Omega^\intercal = \left[v_0 \quad | \quad \Xi\right]$ e,
pela equa��o ~\ref{eq:filtros3}, temos que:

\begin{align}
\dot{v}_0 &= Nv_0 + u \\
\label{eq:dotXi}
\dot{\Xi} &= N\Xi + 
\begin{bmatrix}
-N & -1
\end{bmatrix}
y
\end{align}

Introduzem-se dois novos filtros, para substituir os filtros da
equa��o~\ref{eq:filtros3}:

\begin{align}
\dot{\lambda} &= N\lambda + u \\
\dot{\eta} &= N\eta + y
\end{align}

� f�cil verificar que, para esta planta de segunda ordem sem zeros ($m=0$), $v_0
= \lambda$. 

Agora vamos demonstrar que:
\begin{align}\label{eq:Xi}
\Xi &= -\left[N\eta \quad \eta\right]
\end{align}

Derivando \ref{eq:Xi}, temos:

\begin{align*}
\dot{\Xi} &= -\left[N\dot{\eta} \quad \dot{\eta}\right] \\
&= -\left[N^2\eta + Ny \quad N\eta + y\right]\\
& = -N\left[N\eta \quad \eta\right] + \left[-N \quad -1\right]y\\
& = N\Xi + \left[-N \quad -1\right]y
\end{align*}

E assim chegamos na equa��o \ref{eq:dotXi}. Tamb�m temos a rela��o: 

\begin{align}\label{eq:xi}
\xi &= -N^2\eta
\end{align}

Derivando a equa��o \ref{eq:xi}, obtemos:
\begin{align*}
\dot{\xi} &= -N^2(N\eta + y)\\
&= N(-N^2\eta - y) = N\xi - N^2y 
\end{align*}

E assim chegamos na equa��o \ref{eq:filtros3}. Podemos reescrever a din�mica da
sa�da $y$:

\begin{align}
\dot{y} &= x_2 + \phi^\intercal\theta\\
\nonumber &= k_pv_{0,2} + \xi_2 + \bar{\omega}^\intercal\theta + \epsilon_2 \\
\nonumber \bar{\omega}^\intercal &= 
\begin{bmatrix}
0 & (\Xi_2+\phi_1^\intercal)
\end{bmatrix}
\end{align}